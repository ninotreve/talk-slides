% SIAM Shared Information Template
% This is information that is shared between the main document and any
% supplement. If no supplement is required, then this information can
% be included directly in the main document.


% Packages and macros go here
\usepackage{amssymb}
\usepackage{amsmath}
\usepackage{lipsum}
\usepackage{amsfonts}
\usepackage{graphicx}
\usepackage{commath}
\usepackage{caption}
\usepackage{pifont}
\usepackage{epstopdf}
\usepackage{algorithm,algorithmicx,algpseudocode}
\usepackage{stmaryrd}
\usepackage{multirow}
\usepackage{upgreek}
\usepackage{url}
\usepackage{bbold}
\usepackage{afterpage}
\usepackage[figuresright]{rotating}
\usepackage{subfigure}
\usepackage{ifthen}
\usepackage{xspace}
\usepackage{tikz}
\usepackage{cite}  % not available on Overleaf apparently
\usepackage{latexsym}

\usetikzlibrary{matrix,decorations.pathmorphing,calc,decorations.pathreplacing,topaths,fit,positioning,automata,patterns,arrows,shapes,chains}
\urldef\urlnew\url{https://mvono.github.io}
\ifpdf
  \DeclareGraphicsExtensions{.eps,.pdf,.png,.jpg}
\else
  \DeclareGraphicsExtensions{.eps}
\fi

\usetikzlibrary{arrows.meta}
\tikzset{%
  >={Latex[width=2mm,length=2mm]},
  % Specifications for style of nodes:
            base/.style = {rectangle, rounded corners, draw=black,
                           minimum width=4cm, minimum height=1cm,
                           text centered, font=\sffamily},
  storage/.style = {base, fill=blue!30},
       startstop/.style = {base, fill=red!30},
    activityRuns/.style = {base, fill=green!30},
         process/.style = {base, minimum width=2.5cm, fill=orange!15,
                           font=\ttfamily},
}

% Add a serial/Oxford comma by default.
\newcommand{\creflastconjunction}{, and~}

% Commands
\newcommand{\B}[1]{\mathbf{#1}} % shortcut for \mathbf
\newcommand{\Bs}[1]{\boldsymbol{#1}} % shortcut for \boldsymbol
\newcommand{\pr}[1]{\left(#1\right)} % shortcut for \left(\right)
\newcommand{\br}[1]{\left[#1\right]} % shortcut for \left[\right]
\newcommand{\bbr}[1]{\left\{#1\right\}} % shortcut for \left\{\right\}
\newcommand{\nr}[1]{\left\|#1\right\|} % shortcut for \left\|\right\|
\newcommand{\xmark}{\ding{55}}%
\def\eqsp{\;}


\newcommand{\pc}[1]{\textcolor{blue}{PC: #1}}

\makeatletter
\newcommand{\removelatexerror}{\let\@latex@error\@gobble}
\makeatother

\makeatletter
\newcommand{\thickhline}{%
    \noalign {\ifnum 0=`}\fi \hrule height 1pt
    \futurelet \reserved@a \@xhline
}

\definecolor{green}{rgb}{0.0, 0.5, 0.0}
\definecolor{Burgundy}{RGB}{144,0,32}

\usepackage{pgf}
\usepackage{tikz}
\usetikzlibrary{arrows,automata}
\usetikzlibrary{positioning}
\tikzset{
    state/.style={
           rectangle,
           rounded corners,
           draw=black, very thick,
           minimum height=2em,
           inner sep=2pt,
           text centered,
           },
}

% master block matrix, should use wrappers instead of calling this
% directly
% This should be called within the tikzpicture environment
% \blockmatrix
%  {width}
%  {height}
%  {text}
%  {block_line_color} (can be none --> no line)
%  {block_fill_color} (can be none --> empty fill)
%  {is_diagonal} (true --> true, otherwise --> false)
%  {diagonal_line_color} (ignored if not diagonal) (can be none --> no line)
%  {diagonal_fill_color} (ignored if not diagonal) (can be noneo --> empty fill)
%  {diagonal_offset} (half diagonal width in tikz units)
% Note: colors here are not rgb, they are defined colors
\newcommand{\blockmatrix}[9]{
  \draw[draw=#4,fill=#5] (0,0) rectangle( #1,#2);
  \ifthenelse{\equal{#6}{true}}
  {
    \draw[draw=#7,fill=#8] (0,#2) -- (#9,#2) -- ( #1,#9) -- ( #1,0) -- ( #1 - #9,0) -- (0,#2 -#9) -- cycle;
  }
  {}
  \draw ( #1/2, #2/2) node { #3};
}

% Quick implementation of a tikz right parenthesis
% \rightparen{width}
\newcommand{\rightparen}[1]{
  \begin{tikzpicture} 
    \draw (0,#1/2) arc (0:30:#1);
    \draw (0,#1/2) arc (0:-30:#1);
  \end{tikzpicture}%this comment is necessary
}

% Quick implementation of a tikz left parenthesis
% \leftparen{width}
\newcommand{\leftparen}[1]{
  \begin{tikzpicture} 
    \draw (0,#1/2) arc (180:150:#1);
    \draw (0,#1/2) arc (180:210:#1);
  \end{tikzpicture}%this comment is necessary
}

% Unframed block matrix, "m" prefix to match fbox, mbox
% \blockmatrix[r,g,b]{width}{height}{text}
\newcommand{\mblockmatrix}[4][none]{
  \begin{tikzpicture} 
  \ifthenelse{\equal{#1}{none}}
  {
    \blockmatrix{#2}{#3}{#4}{none}{none}{false}{none}{none}{0.0}
  }
  {
    \definecolor{fillcolor}{rgb}{#1}
    \blockmatrix{#2}{#3}{#4}{none}{fillcolor}{false}{none}{none}{0.0}
  }
  \end{tikzpicture}%this comment is necessary
}

% Framed block matrix
% \fblockmatrix[r,g,b]{width}{height}{text}
\newcommand{\fblockmatrix}[4][none]{
  \begin{tikzpicture} 
  \ifthenelse{\equal{#1}{none}}
  {
    \blockmatrix{#2}{#3}{#4}{black}{none}{false}{none}{none}{0.0}
  }
  {
    \definecolor{fillcolor}{rgb}{#1}
    \blockmatrix{#2}{#3}{#4}{black}{fillcolor}{false}{none}{none}{0.0}
  }
  \end{tikzpicture}%this comment is necessary
}

% Diagonal block matrix
% \dblockmatrix[r,g,b]{width}{height}{text}
\newcommand{\dblockmatrix}[4][none]{
  \begin{tikzpicture} 
  \ifthenelse{\equal{#1}{none}}
  {
    \blockmatrix{#2}{#3}{#4}{black}{none}{true}{black}{none}{0.35cm}
  }
  {
    \definecolor{fillcolor}{rgb}{#1}
    \blockmatrix{#2}{#3}{#4}{black}{none}{true}{black}{fillcolor}{0.35cm}
  }
  \end{tikzpicture}%this comment is necessary
}


% Diagonal block matrix, but exposes diagonal offset
% \diagonalblockmatrix[r,g,b]{width}{height}{text}
\newcommand{\diagonalblockmatrix}[5][none]{
  \begin{tikzpicture} 

  \ifthenelse{\equal{#1}{none}}
  {
    \blockmatrix{#2}{#3}{#4}{black}{none}{true}{black}{none}{#5}
  }
  {
    \definecolor{fillcolor}{rgb}{#1}
    \blockmatrix{#2}{#3}{#4}{black}{none}{true}{black}{fillcolor}{#5}
  }

  \end{tikzpicture}%necessary comment
}

\newcommand{\valignbox}[1]{
  \vtop{\null\hbox{#1}}% necessary comment
}

\newenvironment{blockmatrixtabular}
{% necessary comment
  \begin{tabular}{
  @{}l@{}l@{}l@{}l@{}l@{}l@{}l@{}l@{}l@{}l@{}l@{}l@{}l@{}l@{}l@{}l@{}l@{}l@{}l
  @{}l@{}l@{}l@{}l@{}l@{}l@{}l@{}l@{}l@{}l@{}l@{}l@{}l@{}l@{}l@{}l@{}l@{}l@{}l
  @{}l@{}l@{}l@{}l@{}l@{}l@{}l@{}l@{}l@{}l@{}l@{}l@{}l@{}l@{}l@{}l@{}l@{}l@{}l
  @{}
  }
}
{
  \end{tabular}%necessary comment
}

% Used for creating new theorem and remark environments
\newsiamremark{remark}{Remark}
\newsiamremark{hypothesis}{Hypothesis}
\crefname{hypothesis}{Hypothesis}{Hypotheses}
\newsiamthm{claim}{Claim}
\newsiamremark{example}{Example}

% Sets running headers as well as PDF title and authors
\headers{High-dimensional Gaussian sampling: a review...}{M. Vono, N. Dobigeon, and P. Chainais}

% Title. If the supplement option is on, then "Supplementary Material"
% is automatically inserted before the title.
\title{High-dimensional Gaussian sampling: a review \\and a unifying approach based on a stochastic proximal point algorithm}
%\funding{This work was funded by the Fog Research Institute under contract no.~FRI-454.}}}

% Authors: full names plus addresses.
\author{Maxime Vono\thanks{Lagrange Mathematics and Computing Research Center, Huawei, 75007 Paris, France 
  (\email{maxime.vono@gmail.com}, \urlnew).}
  \and Nicolas Dobigeon\thanks{Univ. of Toulouse, IRIT/INP-ENSEEIHT, Toulouse, France and Institut Universitaire de France (IUF), France
  (\email{nicolas.dobigeon@enseeiht.fr}).}
\and Pierre Chainais\thanks{Univ. Lille, CNRS, Centrale Lille, UMR 9189 CRIStAL, F-59000 Lille, France 
  (\email{pierre.chainais@centralelille.fr}).}}

\usepackage{amsopn}
\DeclareMathOperator{\diag}{diag}
\usepackage{color}
